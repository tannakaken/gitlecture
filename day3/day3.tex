\documentclass[12pt, unicode]{beamer}
\usetheme{CambridgeUS}
\usepackage{luatexja}

\title{Gitによるバージョン管理入門}
\author{田中 健策(株式会社RAKUDO)}
\date[2019/12/18]{第三回}

\begin{document}

\frame{\maketitle}

\begin{frame}{githubのコミュニティ機能}

バージョン管理システムは、コラボレーションを
促進することにより、人を繋げることができます。

自然にgithubはコミュニティを支援する機能を持つようになりました。

それを使いこなすことが、一人ではできない大きな仕事をしたり、
思わぬ出会いを生み出すために重要です。

\end{frame}
\begin{frame}{Wiki}

Wikiはメンバーの誰でも編集できる、コミュニティ用の文書管理ツールです。

githubのリポジトリーにはWikiを用意することができます(これ自体がgitで管理されています)。

この講義用のWikiは\url{https://github.com/tannakaken/nugitlecture2020/wiki}です。

メンバーで共有すべき知識などはここに書きましょう。

今回は、それぞれの担当分と、LaTeXファイルの作成方針を(田中)が書きます。

\end{frame}

\begin{frame}{Issues}

リポジトリーに何か問題が発生した時、
ここに報告して、議論をすることができます。

今回は
\begin{itemize}
\item pullやpushができないなどgitの問題
\item コンパイルができないなどのTeXの問題、
\end{itemize}
をここに投稿すればいいでしょう。

(ただし、これは一般に公開されているので、講義の成績に関する質問などは避けたほうがいいかもしれません)

\end{frame}

\begin{frame}{discussions}

その他discussionsなどの機能もあります。

あまり使ったことはないので便利かはわかりませんが、
作業の担当などで疑問があればここで議論をすることも可能でしょう。

\end{frame}

\begin{frame}{最終目標の確認}

最終的に一枚のPDFになるようなLaTeXファイルを作ってもらいます。

書くことは「何かの問題の解答」です。

適当な大学入試問題などをWebで拾って来ればいいと思います。

作るのが難しそうだったら、簡単な問題の解答を作ればいいです(そこが重要ではないので)。

\end{frame}

\begin{frame}{今回の課題}
リポジトリーのWikiに
「自分はなんの問題の解答を書くか」を書いてください。

Wikiページの右上に「Edit」ボタンがあります。

またそれぞれの作業は、ブランチを分けて行います。

前回作ったテスト用のファイルを自分で掃除して、ブランチを分けて、自分用の作業フォルダを作ってください。

ブランチ名は自分のgithubアカウント名などがいいと思います。

そこに、自分のLaTeXファイルを書いていってください。

ここまでが今回の課題です。

\end{frame}

\begin{frame}[fragile]{次回以降の話}

1/8までに単独でコンパイルできる解答ができているとありがたいです。

それぞれのLaTeXファイルを統合する仕組みは次回以降作ります。

今回はとりあえず自分のLaTeXファイルだけでコンパイルできるようにしてください。

\end{frame}
\begin{frame}{今後の予定}
\begin{itemize}
\item それぞれでコンパイルできるLaTeXファイルを統合してコンパイルする方法
\item プルリクエストによってブランチをマージする
\item github actionによる自動化
\end{itemize}

\end{frame}

\end{document}
