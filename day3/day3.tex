\documentclass[12pt, unicode]{beamer}
\usetheme{CambridgeUS}
\usepackage{luatexja}

\title{Gitによるバージョン管理入門}
\author{田中 健策(株式会社ぺあのしすてむ)}
\date[2019/10/18]{第三回}

\begin{document}

\frame{\maketitle}

\begin{frame}{前回までと今後の方針}

前回まで
\begin{itemize}
\item tortoiseGitの設定のNetwork > Proxyに、名大のProxyサーバーのURLを設定してください。
\item もしまだコラボレーターに登録できていない人がいたら教えてください。
\end{itemize}

今後の方針
\begin{itemize}
\item 分割したLaTeXファイルをまとめて一つのpdfにする方法を考える
\item コンパイルなどの作業をできるだけ自動化する。
\end{itemize}

\end{frame}
\begin{frame}[fragile]{\LaTeX のソース分割 1}

複数人でLaTeXを書くときは、
ファイル分割をする必要がある。

一番簡単なファイル分割は、
\verb|\input|
や
\verb|\include|
を使うものである。

しかしこれには、
単独のファイルでコンパイルできないという欠点がある。

LaTeXのコンパイルはそれなりの時間がかかるので、これは避けたい
(コンパイルして結果を見るためだけに数秒かかるのは、執筆時に多大なストレスになる)。

\end{frame}
\begin{frame}{\LaTeX のソース分割 2}

そこで、自分でマクロを各方法や、用意されたパッケージを使う方法など、
様々な方法が開発されている。

\begin{itemize}
\item 効率的なLaTeXファイル分割術 \url{https://qiita.com/wtsnjp/items/6ba3b8e12514d8a3bd41}
\item 分割したLaTeXファイルをsubfilesを使ってコンパイルする \url{https://qiita.com/sankichi92/items/1e113fcf6cc045eb64f7}
\end{itemize}

今回は、subfilesパッケージを使ってみようと思う。

\end{frame}

\begin{frame}{git hooks}

gitには操作をした時に、自動的に別のプログラムを走らせる\textbf{フック}という仕組みがある。

これを使えば、様々な定型作業(決まり切った作業)を自動化することができる。

ただし使うためには\textbf{シェルスクリプト}などの、コマンドライン・プログラミングの知識が必要になる。

\end{frame}

\begin{frame}{ローカルリポジトリ用のフック}
\begin{itemize}
\item pre-commit commitする前に走る commitを中止できる
\item post-commit commitした後に走る
\item pre-push pushする前に走る pushを中止に知ることができる
\item poist-push pushした後に走る
\item checkout前後やmerge前後やrebase前後に走るフックもある
\end{itemize}

これを使うと、

\begin{itemize}
\item commit前にコンパイルを走らせ、失敗したらcommit中止
\item push前に検査プログラムを走らせ、例えばルールに反していたらpush中止
\end{itemize}

などの自動化ができる。

\end{frame}
\begin{frame}{リモートリポジトリ用のフック}
\begin{itemize}
\item pre-receive pushを受け取った直後に走るフック pushをリモートの側から中止にできる
\item update pushを受け取った時に、ブランチごとに走るフック
\item post-receive push後、全ての処理が終わった後に走るフック
\end{itemize}

これを使うと

\begin{itemize}
\item テストに失敗すればpushを中止
\item masterブランチにpushしたときだけ、ファイルをコンパイルして、そのpdfファイルを自動的にwebページに設定する
\item 全ての処理終了後、変更部分等をメール、slack、Twitterで通知。またCIツールなどに情報を渡すこともできる
\end{itemize}

などの自動化ができる。

\end{frame}
\begin{frame}{Githubの自動化ツール}

gitのフックでできることは限られているが、もっと高度な自動化機能が求められることもある。

ソフトウェア業界では様々な定型作業を自動化していく習慣を\textbf{CI(継続的インテグレーション)}や
\textbf{CD(継続的デプロイ)}と呼ぶ。

Githubは今までCI/CDツールについてはGitlabに遅れをとり、外部のCI/CDツールとの連携が必要であったが、
マイクロソフトによるGithub買収後、マイクロソフトのマネーパワー流入によって、必要なツールは急ビッチで整えられている。

特にCI/CDツールについてGithubは\textbf{Github actions}というツールを昨年リリースし、
今年に入ってそれがCI/CDのツールとしての機能を標準サポートするようになった。

今回はこれを使って、\LaTeX の自動コンパイルと
pdfの自動的な追加を実現してみよう。

\end{frame}

\end{document}
