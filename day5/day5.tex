\documentclass[12pt, unicode]{beamer}
\usetheme{CambridgeUS}
\usepackage{luatexja}

\title{Gitによるバージョン管理入門}
\author{田中 健策(株式会社ぺあのしすてむ)}
\date[2019/10/30]{第五回(最終回)}

\begin{document}

\frame{\maketitle}

\begin{frame}{前回までのあらすじと今回のあらすじ}

前回までのあらすじ
\begin{itemize}
\item プルリクエストを使ったブランチのレビューとマージの自動化と記録
\item Github Actionsを使ったコンパイルとリリースの自動化
\end{itemize}

今日はgitと直接的な関係はないものの、
Github Actionsに絡めて、
\textbf{仮想化}、\textbf{コンテナ化技術}
を少し話します。

\end{frame}

\begin{frame}{仮想化技術とは何か}

\textbf{仮想化(virtualization)}とは、
実際には存在しないものを、まるで存在しているかのように見せかけることである。

コンピュータ技術における主要な仮想化としては、
\textbf{仮想マシン}
などがある。

\vspace{1.0\baselineskip}

仮想マシンとはコンピュータの中に、メモリを持ち、ハードディスクを持ち、画面やキーボードやマウスを持つ「コンピュータ」を丸ごとシミュレートする。

Macマシンの中にWindowsやLinuxなど別のOSがインストールされたマシンをシミュレートできる。

\textbf{Virtualbox}などをインストールすれば誰でも使える。

\end{frame}

\begin{frame}{仮想マシンは何が嬉しいか}
\begin{itemize}
\item 簡単に作れる(本物のコンピュータは簡単に作れない)
\item 簡単にコピーできる(本物のコンピュータは簡単に以下同文)
\item 簡単に捨てられる(以下同文)
\end{itemize}

なので
\begin{itemize}
\item たくさん作れる
\item 実験が簡単にできる
\item 設定などを自動化できる
\item 必要な時に作って終わったら捨てられる
\end{itemize}

\vspace{1.0\baselineskip}

動かしたい環境と同じ環境を手元で作るためなどに使われる。
\end{frame}
\begin{frame}{仮想マシンの弱点}
仮想マシンはコンピュータを丸ごと使うので、色々と\textbf{重い}。

\begin{itemize}
\item 立ち上がりが遅い
\item メモリをたくさん使う
\item 何十GBもの大きなファイルが必要になる
\item 動き自体も遅い
\end{itemize}

これでは例えば本番では使えない。
つまり、
「実際に動かすときの環境を手元に作ること」ではなく、
「作ったときと同じ環境で、別の場所で動かすこと」
はできない。

もっと気軽に作っては壊して、実験し、そのまま持ち運びたい。

そうすれば、作った時と同じ環境で、別の場所でも動かせるし、複数同時に動かすのも簡単になる。

そこで登場するのが\textbf{コンテナ化技術}である。
\end{frame}
\begin{frame}{コンテナ化技術とは何か}

コンテナ化とは、アプリケーションを動かすのに必要なOSの機能や環境だけをうまく仮想化する技術である。

コンピュータを丸ごと仮想化しているわけではないので、軽い。

特に有名なのは\textbf{Docker}というソフトウェアで、
Dockerコンテナというものに必要な環境を入れて動かす。

これでアプリケーションは、どこでもほぼ同じ環境の中で動く。

また既存のDockerコンテナが多数公開されており、
例えば「テスト用のメールサーバが欲しい」が必要なら「それ用のDockerコンテナ」
をダウンロードして起動するだけで、すぐに使える。

また、新しいDockerコンテナを作るための元となるコンテナも多数公開されている。

\end{frame}

\begin{frame}{Github ActionsでDockerを動かす}
Github Actionsでは、Dockerfileというファイルを作成することで、
既存のDockerコンテナをダウンロードし、
調整し、起動することができる。

それによって、複数のactionが同時に起動してもお互いに、
まるで違うコンピュータの上で動いているように、
干渉し合わず動かすことができる。

また、何回作動させても、同じ環境から始まるので、1回目と2回目で結果が変わらない。
これをこの操作は\textbf{べき等性(itempotent)}を持つ、という。
これにより、テストがしやすく、問題も解決しやすい。


\end{frame}
\begin{frame}{なぜ仮想化技術が重要なのか}

Gitと無関係な仮想化技術の話をなぜしたかというと、
これを習得すると、気軽に計算機環境が手に入るからである。

紙と鉛筆が数学の最強の道具であるのは、
気軽に書いては捨てられるから。コンピュータはまだそこまで柔軟ではない。

まだまだ様々な技術の開発が必要ではあるが、
仮想化技術が一つの重要な役割を果たすことは確実である。

今回Github actionsを皆に使ってもらい、Dockerfileを皆に書いてもらうことはできなかったが、
「数学用の精算機を気楽に作っては捨てて実験でき、別の場所で同じ動きを再現できたら便利」
という「お話」は覚えてもらいたい。

というところで、この講義は終わり。

道具こそ人間を人間たらしめてるものの一つなので、道具をうまく使って人生と数学を生きていってください。

\end{frame}


\end{document}
