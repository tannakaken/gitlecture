\documentclass[12pt, unicode]{beamer}
\usetheme{CambridgeUS}
\usepackage{luatexja}

\title{Gitによるバージョン管理入門}
\author{田中 健策(株式会社ぺあのしすてむ)}
\date[2019/10/30]{第五回(最終回)}

\begin{document}

\frame{\maketitle}

\begin{frame}{前回までのあらすじと今回のあらすじ}

前回までのあらすじ
\begin{itemize}
\item プルリクエストを使ったブランチのレビューとマージの自動化と記録
\item Github Actionsを使ったコンパイルとリリースの自動化
\end{itemize}

今回はGitの内部構造といくつかの細かい機能について説明する。

これを知っておくとGitにおけるトラブルシューティングに役に立つからだ。

\end{frame}
\begin{frame}{Gitの内部構造 1}
Gitの内部構造は全て.gitディレクトリにある。

hooksディレクトリには以前説明したgitフックのスクリプトが入っている。

configファイルにはこのリポジトリ独自の設定が入っている。

indexファイルにはgit addされて、これから記録されるファイルが登録されている。

特に重要なのは
\begin{itemize}
\item objectsディレクトリ
\item refsディレクトリ
\item HEADファイル
\end{itemize}
である。

\end{frame}
\begin{frame}{Gitの内部構造 2}

objectsにはgitに記録された全てのコミットやフォルダやファイルの情報が圧縮されて保存されている。
それぞれのobjectにはそのデータのhash値が名前として割り当てられている(フォルダ名とファイル名になっている)。

refsディレクトリには、branchの先頭やtagとして登録されたコミット等のobjectsが収納されている。

HEADファイルは、最新のコミットのobjectが収納している。

git switch(checkout)によってHEADの位置が変わり、その位置のコミットの指し示すフォルダのobjectがgitで管理されたフォルダ(ワーキングツリーと呼ぶ)に、展開される。

\end{frame}
\begin{frame}{Gitの内部構造 3}

git commitをすると、indexに登録されたファイルやフォルダの情報が圧縮されてgitのobjectsに加えられる。
またフォルダやコメント等のコミット情報も圧縮されてobjectsに加えられるが、
その際、現在のHEADがそのコミットの親として登録される。
そして、refsやHEADが動かされて、コミットが完了する。

\end{frame}
\begin{frame}{失われた時を求めて 1}

HEADがbranchの先頭を見ていれば、git commitをしても、branchの先頭のrefが動き、HEADはそのブランチのrefを見ているだけである。
しかし、git switch(checkout)で、branchの途中をHEADが指している時
(このようにブランチの先頭以外をHEADが見ることをdetached HEADと呼ぶ)に、commitすると、
HEADはその加えられたコミットのobjectを見るが、ブランチの先頭のrefは動かない。

その状態で、他のブランチに移動してしまうと、先ほどのコミットはどのrefからも届かないobjectになってしまう。
つまり歴史が消えてしまったのだ。

その他にもgit reset --hard(HEADの位置、インデックス、ワーキングツリーなどが全て戻される)などをして、意図的に歴史を消去することもできる。

\end{frame}
\begin{frame}{失われた時を求めて 2}

しかし消えてしまった記録も、しばらくはobjectsの中に残っている。
そしてかつてHEADで指し示されていたobjectの記録をgit reflogで調べられる。

これを使えば、消えてしまった記録を復元できる。

復元するためのコマンドはgit cherrypickである。
これを使えば、あるコミットの変更を自分のワーキングツリーに反映できる。

\end{frame}
\begin{frame}{変更の一時待避}

gitがファイルやフォルダの情報を保存する仕組みを使えば、
コミットせずにその状態を一時的に記録し、
あとで再現することができる。

それがgit stashである。

急に対応をしなくてはいけない変更などがある場合に便利だ。

コミットしていない変更がある状態で、

git stash

とコマンドすると、変更が退避される。

git stash list

とコマンドすることで退避した作業の一覧が見られ、

git stash apply <stashの番号>

で退避した作業を元に戻すことができる。

\end{frame}
\begin{frame}{git worktree}
.gitの仕組みを使うと、面白いことができる。

例えばgit worktreeコマンドでは、
gitの特定のブランチ用のワークスペースを、現在のワーキングツリーとは別に用意できる。

git worktree add <作業フォルダ> <ブランチ名>

とすると、その場所にそのブランチを展開した作業フォルダができる。
これにより、いちいちブランチを変更しなくても、複数のブランチで同時に作業ができる。

そのフォルダの中には.gitフォルダではなく、.gitファイルがある。

この.gitファイルは、単に元のワーキングツリーの.gitフォルダを参照しているだけのファイルで、
この新しい作業フォルダでgitを操作すると、元の.gitフォルダに影響与えられる。

これにより、混乱することなく、二つの作業フォルダで一つのgitリポジトリを操作することができる。

\end{frame}
\begin{frame}{git submodule 1}
これを応用すると、さらに面白いことができる。
gitリポジトリの中の一部に、別のgitリポジトリを取り込むことができる、
git submoduleという機能である。

git submodule add <gitリポジトリのURL> <フォルダ名>

とすると、現在のリポジトリの中に、
別のgitリポジトリがサブモジュールとして取り込まれる。

そして、現在のgitリポジトリには.gitmodulesというファイルができており、
これは、サブモジュールの.gitディレクトリを参照している。

\end{frame}
\begin{frame}{git submodule 2}

サブモジュールの.gitディレクトリが更新されると、
外側のgitリポジトリはその更新を感知することができ、
.gitmodulesを更新しようとする。

それによって、
外側のgitリポジトリはサブモジュールがどのコミットを指し示せばいいかを記録できるので、

複数のローカルリポジトリで同じサブモジュールのコミットを参照できるのだ
(もしこの仕組みがないと、ローカルリポジトリごとにサブモジュールの別のコミットを取り込んでしまうかもしれない。
そうすると、取り込んだ時期によって、挙動が異なったりすることがありうる)。

これは大きなプロジェクトでは、是非とも使いこなしたい機能である。

\end{frame}
\begin{frame}{便利なログ}

.gitのlogディレクトリには、そのブランチのログが収納されている。
git logは様々なオプションをつけることで、得たい情報を上手く収集することができる。

\begin{itemize}
\item 特定のファイルの変更を追いかける
\item 特定のファイルがいつ消されたかを特定する
\end{itemize}

などができる。

欲しい機能は大体あるので、悩まずに検索してみよう。

またgit bisectというコマンドを使うと、
問題が起こったコミットをbinary searchを使って効率的に探すことができる。

\end{frame}

\end{document}
