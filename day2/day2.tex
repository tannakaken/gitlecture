\documentclass[12pt, unicode]{beamer}
\usetheme{CambridgeUS}
\usepackage{luatexja}

\title{Gitによるバージョン管理入門}
\author{田中 健策(株式会社RAKUDO)}
\date[2020/12/11]{第二回}

\begin{document}

\frame{\maketitle}

\begin{frame}{gitを使ってみよう}
前回の講義の課題をクリアしていれば、\url{https://github.com/tannakaken/nugitlecture2020}のコラボレーターになっているはず。

(講義の資料のソースは\url{https://github.com/tannakaken/gitlecture}においてあります)。

それでは、このリポジトリをローカルにクローンしてみましょう。

CLIを使っている人は
git clone https://github.com/tannakaken/nugitlecture2020.git
でいいです。

GUIを使っている人はそれぞれの環境で試してみましょう。

\end{frame}

\begin{frame}{ファイルをpushしてみましょう}
cloneしたフォルダになんでもいいからファイルを作ってcommitしましょう。
(commitの仕方はwebで調べれば出てきます)。

それをgithubにpushしてみてください(gitにメールアドレスと名前を設定する必要があるかもしれません)。

もし、pushできなければ一度pullしてください。

conflictが起きた、と言われたら次のページを見ましょう。

\end{frame}

\begin{frame}{conflictとは何か?}

リポジトリに違う場所から矛盾する変更を加えると
\textbf{conflict(衝突)}が起こります。

例えばWebブラウザに表示したgithubのページ上とローカルで矛盾する変更を加えれば人工的にconflictが起こせます。

gitはリモートでconflictが起こらないようにします。
なので、pullすべき内容があるときにpushすることはできません(リモートでconflictが起こる可能性があるので)。

pullして、conflictが起こった場合、それを解消するまでpushすることはできません。

\end{frame}
\begin{frame}[fragile]{conflictの例}

例えばgit pullをしてconflictが起こると、ファイルが次のようになります。

\begin{verbatim}
<<<<<<< HEAD
goodbye world
=======
hello world
>>>>>>> 943145c978f2c15e0ee82ec7baae9671dfdec54e
\end{verbatim}

HEADは現在のレポジトリの先頭

下はHEADと衝突しているcommitの番号です。

\end{frame}
\begin{frame}{conflictを解消する}

これをgithubにpushすることはできません。

どちらかいらない方を消すか、
またはより正しい記述に直すかして、
\textbf{conflictを解消}しないといけないのです。。

もし大人数で一つのリポジトリを同時に編集していくと、
誰かが編集するたびにconflictが発生する可能性があります。

これを俗に「\textbf{conflict地獄}」と呼ぶ

\end{frame}
\begin{frame}{branchを分ける}

なので、編集する目的に分けてリポジトリのコピーを作り、
別々に編集します。
それを\textbf{branch(枝)}と言います。

大規模なチームでは、細かくbranchを分けます
(\textbf{branchを切る}という謎の方言を使う人も多い)。

このbranchの分け方の手順を\textbf{flow}といい、いくつか考案されています。

\end{frame}
\begin{frame}{分けたbranchをmergeする}

branchを分けることで、
一つ一つのbranchではそれほど衝突の危険を冒さずに編集できます。

そしてbranchの編集が終わったらそれを、
mainなどの名前のついた中央branchに\textbf{merge(融合)}させます。

その時にconflictが発生すれば、解消します。

これによって、conflict地獄に陥らずに、編集ができるわけです。

また編集途中でも中央branchは汚れていないので、
他の人がそこから別の編集を始めるときにもやりやすくなります。

しかし、branchを分けて長くmergeしないままにすると、
mergeした際にconflictやその他の問題が起きやすくなります。

ですので適切なflowが必要になるわけです。

\end{frame}
\begin{frame}{commitの粒度}

一つのcommitはあまり大きくないことが推奨されます。

何か困ったことがあったあった時に、どのcommitが原因かがわかりやすくしたいです。

またdiffで差分を調べたときにも、見るべき場所が少なくてわかりやすい方が良いです。

一つのcommitは一つの変更、commitのコメントが一言で言えるようにすると良いと言われています。

でも、今回はプログラミングではなく文書なので、正しいやり方は世間でもまだ見つかっていません。

\end{frame}
\begin{frame}{その他gitは奥深い}

gitには過去を改変したり、
過去のグラフを繋げ変えたり、
消えてしまった過去を修復したり、
変更を一時的に記録したり、
と様々な奥深い機能があります。

今回はそこまで行くつもりはませんが
(もし必要になったら、この授業がかなり失敗してる証拠かもしれません)、
もしかしたら将来必要になるかもしれません。

gitは変更を木状に分岐し、それを融合させてグラフ状になるという構造をしています。

それ自体数学的になかなか面白い構造をしているので、
興味があったら調べてみるといいかもしれません。

\end{frame}

\begin{frame}{今週の課題}

\begin{itemize}
\item なんでもいいからgithubにpushしてみよう。
\item もしconflictが起こったら、解消してpushしましょう。
\end{itemize}

\end{frame}

\end{document}
