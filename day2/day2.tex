\documentclass[12pt, unicode]{beamer}
\usetheme{CambridgeUS}
\usepackage{luatexja}

\title{Gitによるバージョン管理入門}
\author{田中 健策(株式会社ぺあのしすてむ)}
\date[2019/10/04]{第二回}

\begin{document}

\frame{\maketitle}

\begin{frame}{前回の訂正・確認等}

色々グダグダですみません。

\begin{itemize}
\item メールアドレスでgithubのコラボレーターに登録できるかと思っていましたが、
メールアドレスが公開されていないとメールアドレスでは登録できないことが判明しました。
すみませんが、\textbf{ユーザー名}でメールを送り直してください。

\item この部屋はどうやらものすごくネットワークが悪いです。
なので、git cloneをする時に、全員一斉に行わずに、
時間差でやる必要があるかもしれません。
ご協力をお願いします。

\item ゴミ箱にゴミが溜まっていて問題を起こすこともあると、
指摘を受けました。
ゴミ箱のゴミは消さないと溜まっていきます。
確認して決しておいてください。

\end{itemize}

\end{frame}

\begin{frame}[fragile]{コンフリクトとは何か?}

リポジトリに違う場所から矛盾する変更を加えると
「\textbf{コンフリクト(衝突)}」が起こる。

github上とローカルで矛盾する変更を加えて、
git pullをすると、ファイルが次のようになる。

\begin{verbatim}
<<<<<<< HEAD
goodbye world
=======
hello world
>>>>>>> 943145c978f2c15e0ee82ec7baae9671dfdec54e
\end{verbatim}

HEADは現在のレポジトリの先頭

下はHEADと衝突しているコミットの番号

\end{frame}
\begin{frame}{コンフリクトを解消する}

これをgithubにpushすることはできない。

どちらかいらない方を消すか、
またはより正しい記述に直すかして、
\textbf{コンフリクトを解消}しないとpushできない。

もし大人数で一つのリポジトリを同時に編集していくと、
誰かが編集するたびにコンフリクトが発生する可能性がある。

これを俗に「\textbf{コンフリクト地獄}」と呼ぶ

\end{frame}
\begin{frame}{ブランチを分ける}

なので、編集する目的に分けてリポジトリのコピーを作り、
別々に編集する。
それを「ブランチ(枝)」と言う。

大規模なチームでは、細かくブランチを分ける
(\textbf{ブランチを切る}という謎の方言を使う人も多い)。

このブランチの分け方の手順を\textbf{フロー}といい、いくつか開発されている。

\end{frame}
\begin{frame}{分けたブランチをマージする}

ブランチを分けることで、
一つ一つのブランチではそれほど衝突の危険を冒さずに編集できる。

そしてブランチの編集が終わったらそれを、
mainなどの名前のついた中央ブランチに\textbf{マージ(融合)}させる。

その時にコンフリクトが発生すれば、解消する。

これによって、コンフリクト地獄に陥らずに、編集ができる。

また編集途中でも中央ブランチは汚れていないので、
他の人がそこから別の編集を始めるときにもやりやすい。

\end{frame}
\begin{frame}{コミットの粒度}

一つのコミットはあまり大きくないことが推奨される。

何か困ったことがあったあった時に、どのコミットが原因かがわかりやすくしたい。

またdiffで差分を調べたときにも、見るべき場所が少なくてわかりやすい方が良い。

一つのコミットは一つの変更、コミットのコメントが一言で言えるようにすると良い。

だが、今回はプログラミングではなく文書なので、正しいやり方は世間でもまだ見つかっていない。

\end{frame}
\begin{frame}{その他gitは奥深い}

gitには過去を改変したり、
過去のグラフを繋げ変えたり、
消えてしまった過去を修復したり、
変更を一時的に記録したり、
と様々な奥深い機能がある。

今回はそこまで行くつもりはないが
(もし必要になったら、この授業がかなり失敗してる証拠かもしれない)、
もしかしたら将来必要になるかもしれない。

gitは変更を木状に分岐し、それを融合させてグラフ状になるという構造をしている。

それ自体数学的になかなか面白い構造をしているので、
興味があったら調べてみるといいかもしれない。

\end{frame}
\end{document}
