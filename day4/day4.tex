\documentclass[12pt, unicode]{beamer}
\usetheme{CambridgeUS}
\usepackage{luatexja}

\title{Gitによるバージョン管理入門}
\author{田中 健策(株式会社ぺあのしすてむ)今日が誕生日!}
\date[2019/10/25]{第四回}

\begin{document}

\frame{\maketitle}

\begin{frame}{前回までのあらすじ}

前回まで
\begin{itemize}
\item 分割したLaTeXファイルを一つにまとめてpdf化する方法を考えた
\item gitのフックの考え方を紹介した。
\item masterブランチにpushするとLaTeXを自動コンパイルするようにしようとした
\end{itemize}

今後の方針
\begin{itemize}
\item メールの一斉送信などを使って、うまくプロジェクトを進行させる。
\end{itemize}

\end{frame}
\begin{frame}{フローとは}

gitでの開発におけるワークフローのベストプラクティクスに名前をつけたもの。

様々なものが開発されている。

有名なものに

\begin{itemize}
\item git flow
\item github flow
\item gitlab flow
\end{itemize}

などがある。

また、それを支援するツールも存在する。

実際にはこれらを見て、自分たちのプロジェクトごとに調整してベストプラクティクスを探していくことになる。

\end{frame}


\end{document}
