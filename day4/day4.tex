\documentclass[12pt, unicode]{beamer}
\usetheme{CambridgeUS}
\usepackage{luatexja}

\title{Gitによるバージョン管理入門}
\author{田中 健策(株式会社RAKUDO)}
\date[2020/12/25]{第四回}

\begin{document}

\frame{\maketitle}

\begin{frame}{前回までのあらすじ}

conflictが起こらないように、ブランチを分けて開発をすることにしました。
しかし、それらを統合しないと完成しません。

そのためには

\begin{itemize}
\item 各ブランチで独立して開発ができて、
\item 統合が楽で(人手を介さず、自動的にできる)
\item 結果がわかりやすい
\end{itemize}

ような手法が必要です。

それを
\begin{itemize}
\item LaTeXのソース分割
\item githubのプルリクエスト
\item github actions
\end{itemize}
によって構築してみましょう。

\end{frame}

\begin{frame}[fragile]{\LaTeX のソース分割 1}

複数人でLaTeXを書くときは、
ファイル分割をする必要です。

一番簡単なファイル分割は、
\verb|\input|
や
\verb|\include|
を使うものです。

しかしこれには、
単独のファイルでコンパイルできないという欠点があります。

LaTeXのコンパイルはそれなりの時間がかかるので、これは避けたいです
(コンパイルして結果を見るためだけに数秒かかるのは、執筆時に多大なストレスになります)。

\end{frame}
\begin{frame}{\LaTeX のソース分割 2}

そこで、自分でマクロを各方法や、パッケージを使う方法など、
様々な方法が開発されています。

\begin{itemize}
\item 効率的なLaTeXファイル分割術 \url{https://qiita.com/wtsnjp/items/6ba3b8e12514d8a3bd41}
\item 分割したLaTeXファイルをsubfilesを使ってコンパイルする \url{https://qiita.com/sankichi92/items/1e113fcf6cc045eb64f7}
\end{itemize}

今回は、二つ目のsubfilesパッケージを使います。
TeX Liveに含まれているが、もし各自の環境に含まれていなければインストールしておいてください。

main.texとintroductionフォルダの中のintroduction.texのファイルがmainブランチにあります。
この2つが書き方の例になっていますので、見ておいてください(次のスライドとWikiにも書き方を書いておきます)。

\end{frame}
\begin{frame}[fragile]{subfilesパッケージの使い方}

まず、統合する方の\LaTeX ファイル(今回はmain.tex)のプリアンプル

\verb|\usepackage{subfiles}|

を書き、本文の挿入したい位置に

\verb|\subfile{自分のフォルダ/自分のファイル}|

を加えます。

そして自分の\LaTeX ファイルのプリアンプルは

\verb|\documentclass[../main]{subfiles}|

だけにしておきます。

必要なプリアンプルはmain.texのプリアンプルに書いておいてください(独自設定を使いすぎるとconflictが起こるので気をつけてください)。

\end{frame}

\begin{frame}{gitignoreについての注意}
gitリポジトリーには普通、生成されるファイルは登録しません。
登録しないファイルのパターンは「.gitignore」というファイルに書いておきます。
今回から、pdfファイルを.gitignoreに加えました。
すでに生成されたpdfファイルをリポジトリーに登録してしまった人は、「git rm」を使って消しておいてください。

図などのためにpdfファイルをリポジトリーに登録する必要がある人は、そのpdfファイル(例えばtanaka/diagram.pdf)を
.gitignoreに

!tanaka/diagramp.pdf

などのように無視しないように設定してください。

詳しくはウェブで.gitignoreについて調べればわかります。


\end{frame}

\begin{frame}{プルリクエスト}

良い仕組みを作っても、
もしメンバーの一部ががおかしなファイルをリポジトリーに加えてしまえば、
他の人までコンパイルできなくなって迷惑を被ります。

そうならないようにする仕組みが\textbf{プルリクエスト}です。

これはgitではなくGithubの機能です。

「レビューをして、問題ないことを確認したらマージ」
という作業をタスク化して、
やりとりを記録できるようにしてくれるのです。

\end{frame}
\begin{frame}{プルリクエストの流れ}

\begin{itemize}
\item Githubのリポジトリの上部のボタンでプルリクエストを作成
\item 担当者がレビューして、修正が不要ならマージされます
\item 修正が必要なら修正を依頼し、修正をpush後またレビュー
\item そもそもプルリクエスト自体が不要ならクローズされます
\end{itemize}

このような流れが関係者にオープンになり、
また記録されます。

またオープンなリポジトリなら、自由にプルリクエストが作成できるので、
自由に開発に参加することができます(もちろん担当者がプルリクエストのレビューを捌けるならではありますが)

\end{frame}

\begin{frame}{今回のプルリクエストの使い方}

\begin{itemize}
\item 定期的にmainブランチを自分のブランチにmergeしておいて、mainの変更を取り入れてください。
\item 自分のブランチのLaTeX作成が終わったら、コンパイルできることを確認します。
\item main.texを編集して、自分の作ったLaTeXファイルがmain.texをコンパイルしたときに取り入れられるようにします。
\item mainブランチへのプルリクエストを作成します。
\item 講師の田中が確認して、mainへmergeします。
\end{itemize}

\end{frame}
\begin{frame}{github actions}

githubなどのgithubホスティングサイトでは、
その性質上さまざまなイベントが起きます。

そのイベントを起点に、さまざまな処理を自動的に走らせるのが、
github actionsです(それ以外にwebhooksも同様の使い方ができます)。

今回はgithub actionsを使って、「mainブランチへの変更があったら、自動的にコンパイルしてpdfをダウンロード可能にする」
という仕組みを作ります。


その仕組み自体は講師の田中が作るが、興味がある方は.githubフォルダ内に関連ファイルがあるので、設定方法を確認してみてください(dockerなどの知識が必要です)。

\end{frame}
\begin{frame}{結果の見方}

mainブランチに変更があると、dockerという仕組みを使って\LaTeX をコンパイルする仮想環境が作られ(結構時間がかかる)、main.texからpdfが生成されます。

生成されたpdfは\url{https://github.com/tannakaken/nugitlecture2020/releases/}からダウンロードできます。

これによって、現在のプロジェクトの進捗が可視化されます。

またコンパイルに失敗すると、管理者(この場合講師の田中)にメールが届きます。これで失敗の可視化もされます。

プログラミングや文章の執筆など、形がなくて進捗が見えにくいプロジェクトにおいて、これは重要です。
\end{frame}

\begin{frame}{今回の課題}

\begin{itemize}
\item 自分の\LaTeX ファイル(なんらかの問題の解答。何の問題に解答したかを記入)を完成させる。
\item 自分のファイルをsubfilesに対応させ、main.texを編集して、自分のファイルだけでも、main.texに統合してでも、どちらでもコンパイルできるようにする。
\item mainブランチへのプルリクエストを作成する。
\end{itemize}
\end{frame}

\end{document}
